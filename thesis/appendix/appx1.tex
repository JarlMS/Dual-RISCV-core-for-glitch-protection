\chapter{Appendix 1 - Code}
\label{app:appx1}

Any source code referenced in the project is shown in this appendix.

\section{hello-world simulation C-code}
\label{app:helloworldC}

\lstset{ 
   language=C,                   % choose the language of the code
   breaklines=true,                % sets automatic line breaking
   breakatwhitespace=false,        % sets if automatic breaks should only happen at whitespace
}
\begin{lstlisting}[caption={A sample C++ code}, label=lst:sample_code]
    /*
**
** Copyright 2020 OpenHW Group
**
** Licensed under the Solderpad Hardware Licence, Version 2.0 (the "License");
** you may not use this file except in compliance with the License.
** You may obtain a copy of the License at
**
**     https://solderpad.org/licenses/
**
** Unless required by applicable law or agreed to in writing, software
** distributed under the License is distributed on an "AS IS" BASIS,
** WITHOUT WARRANTIES OR CONDITIONS OF ANY KIND, either express or implied.
** See the License for the specific language governing permissions and
** limitations under the License.
**
*******************************************************************************
**
** Sanity test for the CV32E40S core.  Reads the MVENDORID, MISA, MARCHID and
**                                     MIMPID CSRs and prints some useful (?)
**                                     messages to stdout.  Will fail if these
**                                     CSRs do not match expected values.
**
*******************************************************************************
*/

#include <stdio.h>
#include <stdint.h>
#include <stdlib.h>

#define EXP_MISA 0x40901104

int main(int argc, char *argv[])
{

    volatile unsigned int misa_rval, mvendorid_rval, marchid_rval, mimpid_rval, mxl;
    volatile          int reserved, tentative, nonstd, user, super;

    mxl = 0; reserved = 0; tentative = 0; nonstd = 0; user = 0; super = 0;

    /* inline assembly: read mvendorid and misa */
    __asm__ volatile("csrr %0, 0xF11" : "=r"(mvendorid_rval));
    __asm__ volatile("csrr %0, 0x301" : "=r"(misa_rval));
    __asm__ volatile("csrr %0, 0xF12" : "=r"(marchid_rval));
    __asm__ volatile("csrr %0, 0xF13" : "=r"(mimpid_rval));

    /* Check MVENDORID CSR: 0x602 is the value assigned by JEDEC to the OpenHW Group */
    if (mvendorid_rval != 0x00000602) {
      printf("\tERROR: CSR MVENDORID reads as 0x%x - should be 0x00000602 for the OpenHW Group.\n\n", mvendorid_rval);
      return EXIT_FAILURE;
    }

    /* Check MISA CSR: if its zero, it might not be implemented at all */
    if (misa_rval != EXP_MISA) {
      printf("\tERROR: CSR MISA reads as 0x%x - should be 0x%x for this release of CV32E40S!\n\n", misa_rval, EXP_MISA);
      return EXIT_FAILURE;
    }

    /* Check MARCHID CSR: 0x15 is the value assigned by the RISC-V Foundation to CV32E40S */
    if (marchid_rval != 0x15) {
      printf("\tERROR: CSR MARCHID reads as 0x%x - should be 0x00000015 for CV32E40S.\n\n", marchid_rval);
      return EXIT_FAILURE;
    }

    /* Check MIMPID CSR: 0x0 is the value assigned by the OpenHW Group to the first release of CV32E40S */
    if (mimpid_rval != 0x00000000) {
      printf("\tERROR: CSR MIMPID reads as 0x%x - should be 0x00000000 for this release of CV32E40S.\n\n", mimpid_rval);
      return EXIT_FAILURE;
    }

    /* Print a banner to stdout and interpret MISA CSR */
    printf("\nHELLO WORLD!!!\n");
    printf("This is the OpenHW Group CV32E40S CORE-V processor core.\n");
    printf("CV32E40S is a RISC-V ISA compliant core with the following attributes:\n");
    printf("\tmvendorid = 0x%x\n", mvendorid_rval);
    printf("\tmarchid   = 0x%x\n", marchid_rval);
    printf("\tmimpid    = 0x%x\n", mimpid_rval);
    printf("\tmisa      = 0x%x\n", misa_rval);
    mxl = ((misa_rval & 0xC0000000) >> 30); // MXL == MISA[31:30]
    switch (mxl) {
      case 0:  printf("\tERROR: MXL cannot be zero!\n");
               return EXIT_FAILURE;
               break;
      case 1:  printf("\tXLEN is 32-bits\n");
               break;
      case 2:  printf("\tXLEN is 64-bits\n");
               break;
      case 3:  printf("\tXLEN is 128-bits\n");
               break;
      default: printf("\tERROR: mxl (%0d) not in 0..3, your code is broken!\n", mxl);
               return EXIT_FAILURE;
    }

    printf("\tSupported Instructions Extensions: ");
    if ((misa_rval >> 25) & 0x00000001) ++reserved;
    if ((misa_rval >> 24) & 0x00000001) ++reserved;
    if ((misa_rval >> 23) & 0x00000001) {
      printf("X");
      ++nonstd;
    }
    if ((misa_rval >> 22) & 0x00000001) ++reserved;
    if ((misa_rval >> 21) & 0x00000001) ++tentative;
    if ((misa_rval >> 20) & 0x00000001) ++user;
    if ((misa_rval >> 19) & 0x00000001) ++tentative;
    if ((misa_rval >> 18) & 0x00000001) ++super;
    if ((misa_rval >> 17) & 0x00000001) ++reserved;
    if ((misa_rval >> 16) & 0x00000001) printf("Q");
    if ((misa_rval >> 15) & 0x00000001) ++tentative;
    if ((misa_rval >> 14) & 0x00000001) ++reserved;
    if ((misa_rval >> 13) & 0x00000001) printf("N");
    if ((misa_rval >> 12) & 0x00000001) printf("M");
    if ((misa_rval >> 11) & 0x00000001) ++tentative;
    if ((misa_rval >> 10) & 0x00000001) ++reserved;
    if ((misa_rval >>  9) & 0x00000001) printf("J");
    if ((misa_rval >>  8) & 0x00000001) printf("I");
    if ((misa_rval >>  7) & 0x00000001) printf("H");
    if ((misa_rval >>  6) & 0x00000001) printf("G");
    if ((misa_rval >>  5) & 0x00000001) printf("F");
    if ((misa_rval >>  4) & 0x00000001) printf("E");
    if ((misa_rval >>  3) & 0x00000001) printf("D");
    if ((misa_rval >>  2) & 0x00000001) printf("C");
    if ((misa_rval >>  1) & 0x00000001) printf("B");
    if ((misa_rval      ) & 0x00000001) printf("A");
    printf("\n");
    if (super) {
      printf("\tThis machine supports SUPERVISOR mode.\n");
    }
    if (user) {
      printf("\tThis machine supports USER mode.\n");
    }
    if (nonstd) {
      printf("\tThis machine supports non-standard instructions.\n");
    }
    if (tentative) {
      printf("\tWARNING: %0d tentative instruction extensions are defined!\n", tentative);
    }
    if (reserved) {
      printf("\tERROR: %0d reserved instruction extensions are defined!\n\n", reserved);
      return EXIT_FAILURE;
    }
    else {
      printf("\n");
      return EXIT_SUCCESS;
    }
}

\end{lstlisting}

\subsubsection{Subsection}
Write something else here.