%Start with a Clear Heading: Begin your report with a heading that clearly states "Background" or "Introduction" to let readers know where to find this information.

%Provide an Overview: Begin by providing a brief overview of the project, including its title and purpose. You can mention what the project aims to achieve or the problem it intends to address.

%Contextualize the Project:

%Explain the broader context in which the project exists. Discuss any relevant industry trends, challenges, or issues that the project is responding to.
%Mention any external factors or events that have prompted the need for the project.
%If the project is part of a larger initiative or program, briefly introduce that program and explain how your project fits into it.
%Historical Background:

%Provide historical information if relevant. Explain how this project came into consideration, including any past efforts or developments leading up to it.
%If there are key milestones or events that are crucial to understanding the project's origins, mention them.
%State the Problem or Opportunity:

%Clearly state the problem or opportunity your project aims to address. Use data or facts to support your claims.
%If applicable, describe the significance of the problem or opportunity, such as its impact on the organization, stakeholders, or the community.
%Objective and Scope:

%Outline the specific objectives and scope of the project. What are you trying to achieve, and what are the boundaries or limitations of the project?
%Explain why these objectives are important and how they relate to solving the identified problem.
%Review of Relevant Literature:

%If there's existing research or literature relevant to your project, briefly mention it. This shows that you've conducted a literature review and are building upon prior knowledge.
%Highlight any gaps or areas where your project adds value or addresses shortcomings in existing research or practices.
%Cite Sources: Make sure to provide proper citations for any data, statistics, or information you use in this section.

%Keep it Concise and Engaging: While providing all necessary information, avoid making the background section overly long. Keep it engaging and focused on the most crucial information.

%Transition to the Next Sections: Conclude the background section by smoothly transitioning into the next sections of your report, such as the objectives, methodology, or project plan.

%Remember to tailor the background section to your specific project and audience. Your goal is to provide a comprehensive but concise overview that sets the stage for the reader to understand the significance and context of your project.

\chapter{Background}
\label{chap3}

Glitch attacks can be divided into two main categories: invasive (e.g., decapsulating the chip) and non-invasive attacks (e.g., electromagnetic fault injection, voltage- and clock-glitching)\cite{glitchresistor}. Often times software and firmware security measures can only protect against non-invasive glitches as protecting from invasive glitches necessarily require hardware modifications. 

Glitch attacks have over recent years become a more common and greater threat. Due to the nature of how these attacks are carried out, they are often a very affordable way to exploit hardware. Attackers have easy access to open source resources like the the 'ChipWhisperer Lite' which allows for easy access to hardware glitching tools as well as side channel analysis\cite{chipWhisperer}. In addition to this, FPGAs can be used to inject precisely timed faults as long as an attacker has access to the supply voltage or clock on a chip\cite{hole_in_soc}. 

\section{What are glitch attacks / fault injection}
\label{sec:glitch}

Glitch attacks can be performed in several different ways, 

\subsection{Power glitching}

\subsection{Clock glitching}

\subsection{Electromagnetic fault injection}

\subsection{Invasive / non-invasive glitching}

\section{What can glitch attacks do and how do they make a chip succeptable to attacks?}

\section{Introduction to RISC-V}
\section{The need for glitch protection}
\section{State of the Art}
\section{Previous Approaches to glitch protection}
\section{CV32E40S}
\label{sec:cv32}
\subsection{Xsecure Extension}
\subsubsection{Hardened PC}
\subsubsection{Hardened CSR}
\section{Limitations of glitch protection}
\label{sec:limits}

\begin{itemize}
    \item High logic cost 
    \item Slower execution of certain pipeline stages
    \item Lower throughput 
    \item A typical way of doing fault injection is though EMFI attacks which target larger components like caches or other memory components.
    How can we prevent these? Comparing the state of the entire chache is not viable, and thus we must allow a fault to persist in 
    memory untill it is read or used in one of the cores.
\end{itemize}
\section{The dual core proposition}
\section{Rationale for this study}

