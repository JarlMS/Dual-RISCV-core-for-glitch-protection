\chapter{Introduction}
\label{intro} 

Modern integrated systems and other programmable electronics very often have several safety-measures built in to avoid unintended behaviour, whether it be accidental or from an external source. These safety measures are often only handled on the software or firmware level, but for a long time hardware induced faults, so called glitches, have been a concern that has gained less focus even though they are capable of bypassing software security measures given the right attacker. Interestingly, despite the relevance of glitch protection, it only started gaining attention after practical glitch attacks first made their appearance in 2002 through an optical fault attack\cite{trouchkine2019fault}. 

In the world of computer architecture, many aspects remain closely guarded secrets. This secrecy extends to preventive measures against such attacks, making them often inaccessible to the public. Over recent years there has been a large shift away from this secrecy as more and more researchers and companies use the \textit{RISC-V} instruction set architecture (ISA). This architecture is open source, allowing anyone with the right knowledge and motivation to create their own central processing unit (CPU). One group of people doing this is the \textit{OpenHW Group} who have developed several publicly available RISC-V CPUs over recent years. One of these is the \textit{CV32E40S}, which is a core mainly focused on securing against glitch- and side-channel attacks\cite{cv32e40s_manual}.

While the \textit{CV32E40S} has a proven to be able to detect glitch attacks, it comes at a cost in performance. Often the methods used to protect against attacks are complex and increase the execution time and power needed for the CPU to perform its given tasks. This project looks at the possibility of removing some of the security features of the \textit{CV32E40S} in favour of running two CPUs in a\textit{Dual-Core Lockstep} mechanism instead to increase throughput and simplicity. 

\section{Sustainability}
