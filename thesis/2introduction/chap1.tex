\chapter{Introduction}
\label{intro} 

Today micro-controllers and other embedded electronics are in almost everything we use. Because of this, ensuring the secure operation of these devices is critical. While much attention is directed towards software and firmware security, hardware security often goes overlooked. Particularly, hardware fault-injection (glitching) has repeatedly demonstrated its capability to bypass conventional security measures. Examples include bypassing cryptographic signature validation of firmware binaries \cite{hole_in_soc}, re-enabling hardware debug functionality on production/fused processors\cite{reenable_debug} and bypassing input validation of data that crosses trust boundaries between privilege levels\cite{qualcom}. To stop these glitch attacks, a number of countermeasures can be introduced. This project looks at the ones implemented by the OpenHW Group for their 'CV32E40S' RISC-V core\cite{cv32e40s_manual}, and how these security measures can be avoided entirely by instead running a dual-core RISC-V setup with lockstep. 

