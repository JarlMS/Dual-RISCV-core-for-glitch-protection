\chapter{Conclusion \& Future Work}
\label{chap6}

In this project we have described what glitch attacks are, how they can be performed and ways of protecting systems against them. The main goal of this project was to attempt to implement a new way of doing glitch protection though redundant operation by having two RISC-V cores running in a Lockstep setup. The core used for this is the CV32E40S core made by the OpenHW group. This core already has built in security features that were shown to add area, power usage and execution time. By attempting to glitch both cores using test described in \autoref{tab:instr_skip_test} and \autoref{tab:coverage_test} it was shown that the Dual-Core Lockstep setup outperforms the single core when it comes to glitch detection. This increased glitch detection does however come at the cost of increased area and power usage as shown in \autoref{tab:ppa_results}. 

The work in this project shows that using the Dual-Core Lockstep setup is advantageous for systems that require a wide coverage for glitch protection. The implementation was shown to be simple to implement with an already existing core. However, there are ways that this implementation can be optimized to reduce area and power usage. This would make it an even more attractive option compared to the existing solution. Future work should for example look into optimizations such as sharing a register file between the cores to reduce area. In addition to this, future work should look into testing the Dual-Core Lockstep setup on an FPGA. This way it is possible to test the setups resilience to real world attacks such as voltage glitching, clock glitching and EMFI. 