\chapter{Conclusion \& Future Work}
\label{chap6}

In this project we have described what glitch attacks are, how they can be performed and ways of protecting systems against them. The main goal of this project was to investigate the viability of the CV32E40DC core. This core implements a new way of doing glitch protection though redundant operation by having two RISC-V cores running in a lockstep mechanism. The work in this project was based on the CV32E40S core\cite{cv32e40s_github} made by the OpenHW group\cite{cv32e40s_manual}. This core already has built in security features that were shown to add area, power consumption and execution time. By attempting to glitch both cores using tests described in \autoref{tab:instr_skip_test} and \autoref{tab:coverage_test} it was shown that the CV32E40DC outperforms the CV32E40S when it comes to detecting glitch attacks. This improved detection does however come at the cost of increased area and power consumption as well as a negligible increase in performance as shown in \autoref{fig:area_power_plot}. 

The results presented in \autoref{chap5} suggest that using the CV32E40DC is advantageous in systems that require a broader range of fault coverage. The concept behind the implementation of this core was shown in \autoref{sec:dualcore} to be simple when working with an already existing core. However, there are ways that this implementation can be optimized to reduce area and power consumption. This would make the CV32E40DC a more viable alternative to the pre-existing CV32E40S. 

Future work on this project should look into optimizations such as a shared register file between the cores to reduce area. As explained in \autoref{sec:xsecure}, we were only able to remove the PCH feature from the pre-existing core. Because of this, further work needs investigate ways of removing and replacing the CSRH and ECC features as well. Lastly, future research should look into testing the CV32E40DC on physical hardware by running it on an FPGA. This way it is possible to test the cores resilience to real world attacks such as voltage glitching, clock glitching and EMFI. 