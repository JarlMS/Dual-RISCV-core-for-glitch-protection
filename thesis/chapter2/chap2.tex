\chapter{Title: Chapter 2}
\label{chap2}
In order to understand how ...

\section{Section}
\label{sec2}
Some useful information is described in~\cite{powerbook} and~\cite{convdef}.

A signal, $ x(t) $, is defined as
\begin{equation}
\label{eq:cwttrans}
	T(a,b) = \frac{1}{\sqrt{a}}\int_{-\infty}^{+\infty}x(t)\psi^{\ast}(\dfrac{t-b}{a})\delta t,
\end{equation} where $ \psi^{\ast}(t) $ is ...

Certain mathematical criteria has to be satisfied in order to ...:
\begin{enumerate}
	\item Item 1
	\item Item 1
\end{enumerate} 

Table~\ref{tab:tab1} is shown below.

\begin{table}[!ht]
\centering
\begin{tabular}{|c|c|}\hline
\multicolumn{2}{|c|}{\emph{Description 1 }} \\ %\hline \hline
\multicolumn{2}{|c|}{\emph{Description 2}} \\ \hline \hline
xxx & yyy \\ \hline
111 & 222 \\ \hline
\end{tabular}
\caption[Short caption]{Long caption}
\label{tab:tab1}
\end{table}

In Algorithm~\ref{alg:modexp1}, ...

\begin{algorithm}[!ht]
\caption{Algorithm caption}\label{alg:modexp1}
\small
\begin{algorithmic}[1]
\Procedure{ModExp}{$M, e, n, k$}{$=M^e\mod n$}
 \State $M_m \gets M\cdot  r \mod n$
 \State $X_m \gets 1\cdot  r \mod n$
 \For{$i\gets k-1, 0$}
  \State $X_m \gets \operatorname{MonPro}(X_m, X_m, n, k)$
  \If{$e_i = 1$}
   \State $X_m \gets \operatorname{MonPro}(M_m, X_m, n, k)$
  \EndIf
 \EndFor
 \State $X \gets \operatorname{MonPro}(X_m, 1, n, k)$
 \State \textbf{return} $X$
\EndProcedure
\end{algorithmic}
% ModExp(M, e, n, k)
% Input:  M, e, n, k
% Output: M^e mod n
% Mm := M*r mod n
% Xm := 1*r mod n
% for i = k-1 downto 0 do
%  	Xm := MonPro(Xm, Xm, n, k)
%  	if (e_i = 1) then Xm := MonPro(Mm, Xm, n, k)
% X := MonPro(Xm, 1, n, k)
% return X
\normalsize
\end{algorithm}
