\chapter{Abstract}

% This project aims at using the CV32E40S RISC-V CPU in a dual-core lockstep mechanism as a way of doing glitch protection. The purpose of this research is to determine whether this simple measure can replace pre-existing and much more complex security features. By proving the effectiveness of this new hardware architecture, it paves the way for developers to simplify their designs and speed up the development of future CPUs. In addition, this project aims at broadening the public knowledge about secure CPU design, as this topic is largely kept a secret by the major players in the industry. 

% To address this, all research was done on an open-source RISC-V core. This core contains a lot of complex security features, which were found to add a somewhat significant amount of resource usage. By simply disabling one of the many security features the overall performance of the system could be increased by 5\%. To compare the quality of the dual-core setup with the existing setup, they were both subjected to different types of simulated glitch attacks while running a simple test program. From these tests we found that the dual-core setup was just as capable of detecting errors, and even surpassed the existing setup in some cases. This means that it is a simple and viable alternative for researchers aiming to secure their hardware against glitch attacks. 

% To advance this research, the next phase involves fabricating an actual chip based on the proposed architecture and assessing its resilience against various real-world glitch attacks. Specific focus will be given to electromagnetic fault injection, voltage manipulation, and clock glitching techniques, which pose substantial threats to modern computing systems. This comprehensive evaluation will provide important insights into the practical viability of the dual-core lock-step mechanism as a security solution for RISC-V cores.

This project is done in collaboration with Silicon Laboratories Norway AS and investigates a new way of doing glitch protection in RISC-V cores. For this purpose we introduce the CV32E40DC. This is a modified version of the CV32E40S core made by the OpenHW group\cite{cv32e40s_manual} leveraging a dual-core lockstep mechanism. We demonstrate how this core, while simple to implement, offers increased system robustness and fault detection coverage.  

The CV32E40DC is compared to the power, performance and area (PPA) benchmarks set by the CV32E40S. After synthesis of both cores it was found that the CV32E40DC occupies $122.73\%$ more area and consumes $189.34\%$ more power than the benchmark. The impact to performance was shown to be negligible as this increased by between $0.05\%$ and $0.32\%$. This could not be improved as performance is limited by side-channel attack prevention features which can not be replaced by a dual-core lockstep mechanism. In addition to this both cores were subjected to tests meant to simulate possible glitch attacks. The quality of a core in this project was evaluated based on its ability to detect these simulated attacks. From these tests we found that the CV32E40DC outperformed the CV32E40S and was able to detect all faults injected to the system. 

These findings suggest that with optimizations to bring the PPA metrics of the CV32E40DC closer to those of the CV32E40S, it can be a viable alternative for systems requiring a broader range of fault coverage. 


\chapter{Preface}

I would like to thank my technical supervisor Øystein Knauserud for giving me excellent guidance and advice whenever I was stuck or did not understand the intricacies of the system I was working on. In addition I would like to thank my academic supervisor Thomas Tybell for providing me with good feedback and inspiration for how to write this report. 

